\documentclass[12pt, a4paper]{report}

\usepackage{polyglossia}
\usepackage{amsmath} % Advanced math typesetting
\usepackage[utf8]{inputenc} % Unicode support (Umlauts etc.)
\usepackage{hyperref} % Add a link to your document
\usepackage{graphicx} % Add pictures to your document
\usepackage{listings} % Source code formatting and highlighting
\usepackage{biblatex}
\usepackage{titlesec}
\nocite{*}


\usepackage{color}
\definecolor{lightgray}{rgb}{.9,.9,.9}
\definecolor{darkgray}{rgb}{.4,.4,.4}
\definecolor{purple}{rgb}{0.65, 0.12, 0.82}

\lstdefinelanguage{JavaScript}{
  keywords={typeof, new, true, false, catch, function, return, null, catch, switch, var, if, in, while, do, else, case, break},
  keywordstyle=\color{blue}\bfseries,
  ndkeywords={class, export, boolean, throw, implements, import, this},
  ndkeywordstyle=\color{darkgray}\bfseries,
  identifierstyle=\color{black},
  sensitive=false,
  comment=[l]{//},
  morecomment=[s]{/*}{*/},
  commentstyle=\color{purple}\ttfamily,
  stringstyle=\color{red}\ttfamily,
  morestring=[b]',
  morestring=[b]"
}

\lstset{
   language=JavaScript,
   backgroundcolor=\color{lightgray},
   extendedchars=true,
   basicstyle=\footnotesize\ttfamily,
   showstringspaces=false,
   showspaces=false,
   numbers=left,
   numberstyle=\footnotesize,
   numbersep=9pt,
   tabsize=2,
   breaklines=true,
   showtabs=false,
   captionpos=b
}

\setdefaultlanguage{romanian}
\setotherlanguages{english}

% Indentează și primul paragraf al fiecărei noi secțiuni
\SetLanguageKeys{romanian}{indentfirst=true}

\titleformat{\section}
  {\bfseries} % format
  {}                % label
  {0pt}             % sep
  {\huge}

\titleformat{\subsection}
  {\bfseries} 
  {}                % label
  {0pt}             % sep
  {}           % before-code

\addbibresource{bibl.bib}
\begin{document}

\author{Dragos Bahrim}
\title{SQL Injection}
\date{\today{}}
\maketitle{}
\tableofcontents
\section{Abstract}

Această lucrare încearcă să documenteze detalii, scenarii de execuție și metode de prevenire pentru atacurile de tipul SQL Injection.


\section{Introducere}

SQL Injection reprezintă un tip de atac care se bazează pe faptul că indiferent de 
arhitectura aplicației și de baza de date folosită, pentru a obține date trebuie
rulată o cerere care uneori cere date de intrare de la atacator, care datorită lipsei măsurilor
de securizare oferă mai mult aces utilizatorului decât ar trebui.

Pentru a răspundere la cererile complexe la care ne așteptăm de la bazele de date, structura acestor cereri
este foarte versatilă, din acest motiv dacă nu sunt protejate corespunzător un atacator
poate să folosească cererile scrise incorect pentru a schimba comportamentul sistemului într-un mod nedorit.

\section{Scenarii}
Cel mai simplu mod în care acest tip de atac se reproduce este reprezentat de nevalidarea
datelor pe care utilizatorul le introduce.

Considerăm scenariul extrem de ireal în care unui utilizator îi se cere username-ul și parola pentru
a se autentifica, ca și test, împotriva oricărei recomandare de salvare a datelor confidențiale, 
parola este păstrată în text simplu în baza de date, astfel pentru a ne autentifica trebuie doar să 
căutăm după nume și parolă.

\lstset{language=SQL}
\begin{lstlisting}
  SELECT *
  FROM dbahrim.app_users
  WHERE username = '<VALUE>' and userpassword = '<VALUE>';
\end{lstlisting}


Codul pentru server-ul care face această verificare poate fi următorul

\lstset{language=JavaScript}
\begin{lstlisting}
app.post('/login', async (req, res) => {
    const username = req.body.username
    const password = req.body.password

    try {
        const result = await connection.execute(
            `SELECT *
                FROM dbahrim.app_users
                WHERE username = '${username}' and userpassword = '${password}'`,
        );

        if (result.rows.length > 0) {
            res.end("Success");
        } else {
            res.end("Failure");
        }
    } catch (e) {
        res.json(e).end()
    }
})
\end{lstlisting}

Codul va face un simplu SELECT, dacă se găsește un row cu detaliile 
introduse este considerată o operațiune cu succes.

Întrucât nicio dată de intrare nu este validată, un atacator poate să dea ce date vrea el, 
deci poate introduce caractere speciale pentru baza de date ca să manipuleze request-ul și să treacă de validare.

De exemplu

\lstset{language=bash}
\begin{lstlisting}
curl --request POST \
  --url http://localhost:3000/login \
  --header 'content-type: application/json' \
  --data '{
  "username": "'\'' or 1=1--",
  "password": "dfgdfgdfgfdgfdgdfg"
}'
\end{lstlisting}

Codul se traduce in script SQL în

\lstset{language=SQL}
\begin{lstlisting}
  SELECT *
  FROM dbahrim.app_users
  WHERE username = ''\'' or 1=1-- and userpassword = '<VALUE>';
\end{lstlisting}

Astfel, atacatorul poate să comenteze o porțiune de verificare prin folosirea
`--` și să adauge o condiție care mereu va fi adevarată `1=1`.

De asemenea, putem folosii SQL Injection pentru a face rost de informații
specifice bazei de date, ca după să putem găsii alte exploit-uri.
Considerăm următorea porțiune de cod care ar fi returnat username-ul unui utilizator.

\lstset{language=JavaScript}
\begin{lstlisting}
app.get('/users/:id', async (req, res) => {
    const id = req.params.id
    try {
        const query = `SELECT username
                FROM dbahrim.app_users
                WHERE id_user = '${id}'`
        console.log(query)
        const result = await connection.execute(
            query,
        );
        res.json(result.rows).end()
    } catch (e) {
        res.json(e).end()
    }
})
\end{lstlisting}

Un atacator poate folosii acest tip de cerere pentru a folosii UNION pentru a
extrage mai mult informații. De exemplu:

\label{exemplush1}
\lstset{language=bash}
\begin{lstlisting}
curl --request GET \
  --url 'http://localhost:3000/users/1%27%2520UNION%2520select%2520banner%2520from%2520v$version%2520--'
\end{lstlisting}

Ce va întoarce:

\begin{lstlisting}
[
  {
    "USERNAME": "user1"
  },
  {
    "USERNAME": "Oracle Database 23ai Free Release 23.0.0.0.0 - Develop, Learn, and Run for Free"
  }
]
\end{lstlisting}

\section{Tehnici de protejare}
\subsection{Client}

Prin client, înțelegem orice aplicație al cărei scop este preluarea date de la 
utilizator și trimiterea către un server, fie ca este o aplicație web, API sau 
de mobil.

Principalul scop este prelucrarea datelor înainte de a ajuns la următorul procesator.
Astfel, putem să impunem un anumit format pentru textele introduse. De exemplu, dacă
avem un câmp pentru username într-un formular, folosim regex pentru a limita tipul de caractere pe care să le introducem.


\lstset{language=html}
\begin{lstlisting}
<input type="text" name="username" pattern="[a-zA-Z0-9]+" required>
\end{lstlisting}

Putem să encodăm caracterele speciale care sunt problematice și să le reînlocuim pe server (de exemplu \& cu \&ampt).

Dacă folosim în formulare câmpuri hidden folosite pentru a găsii elementul
pe care vrem să îl modificăm ulterior, acesta trebuie limitat pentru a previne oferirea 
de informații adiționale despre structura bazei de date.

\subsection{Server}

Prin server, înțelegem orice serviciu ce este folosit, fie că este cel cu care comunică 
clientul, fie un microserviciu intern.

Deoarece clientul poate să scoată validările, toate câmpurile validate pe client, trebuie revalidate și în codul de server, inclusiv dacă este un server care comunică doar cu alte servere.

Fiecare valoare ce vine de la utilizator va trece printr-o serie de funcții sanitaziars,
de exemplu pentru Node.JS putem folosii \url{https://github.com/pocketly/node-sanitize}
care ne vor ajuta în îndepărtarea caracterelor care ne-ar compromite cererile SQL. Pentru exemplul 
\hyperref[exemplush1]{anterior} putem sa modificam cererea de pe server pentru a sanitiza input-ul si sa eliminam lucrurile care par problematice.

\lstset{language=JavaScript}
\begin{lstlisting}
  app.get('/users/:id', async (req, res) => {
    const id = req.params.id
    console.log(id) // va intoarce 1' UNION select banner from v$version --
    // ...
  });

  app.get('/users2/:id', require('sanitize').middleware, async (req, res) => {
    const id = req.paramInt("id")
    console.log(id) // va intoarce - 1
    // ..
  });

  app.get('/users3/:id', require('sanitize').middleware, async (req, res) => {
    const id = req.paramString("id")
    console.log(id) // - 1' UNION select banner from v$version --
  });
\end{lstlisting}

In ultimul exemplu, desi folosim acesti sanitizers, nu ne protejeaza atunci cand parsam ca string, caracterele problematice inca ramanand. 
Pentru asa putem folosii bind variables (in Java, Prepared Statements) 
pentru a prevenii modificarea query-ului SQL atunci când adăugăm valorile de la utilizator.

\lstset{language=JavaScript}
\begin{lstlisting}
app.post('/login', async (req, res) => {
  const username = req.body.username
  const password = req.body.password

  try {
      const result = await connection.execute(
          `SELECT *
            FROM dbahrim.app_users
            WHERE username = :username and userpassword = :password`,
          { username, password }
      );

      if (result.rows.length > 0) {
          res.end("Success");
      } else {
          res.end("Failure");
      }
  } catch (e) {
      res.json(e).end()
  }
})
\end{lstlisting}

Folosirea ORM-urilor precum Sequelize pentru Node.JS ne permite să fim mai 
relaxați cu validarea dateloror de intrare, Întrucâtt ne aștepăm ca aceste tool-uri să 
fie scrise cu gândul că vor fii folosite de persoane care nu au la fel de multă 
experiență și vor avea suficiente protecții împotriva SQL Injection.

Pentru a folosii Sequelize declarăm un model ce conține informațiile unui tabel din baza noastră de date.

\lstset{language=JavaScript}
\begin{lstlisting}
const AppUsers = sequelize.define(
  'APP_USERS',
  {
      id_user: {
          type: DataTypes.INTEGER,
          primaryKey: true,
      },
      username: {
          type: DataTypes.STRING,
      },
      userpassword: {
          type: DataTypes.STRING,
      },
  }
);
\end{lstlisting}

Prin acest model vom interacționa cu baza de date, Sequelize generând codul SQL.

\lstset{language=JavaScript}
\begin{lstlisting}
app.get('/users4/:id', require('sanitize').middleware, async (req, res) => {
  const id = req.paramString("id")

  console.log(await AppUsers.findAll({where: {
      id_user: id
  }}));
})
\end{lstlisting}

Se va genera

\lstset{language=JavaScript}
\begin{lstlisting}
Executing (default): SELECT "id_user", "username", "userpassword" FROM "APP_USERS"  "APP_USERS" WHERE "APP_USERS"."id_user" = '1'' UNION select banner from v$version --';
\end{lstlisting}

Observație! Query-ul este gresit datorita unei incompatibilități între versiunea de Sequelize, Oracle.DB și laptop-ul meu.

Ascunderea mesajelor va permite păstrarea informațiilor despre baza de date ascunse,
astfel mesajele de eroare vor trebui să fie cât mai generice.

\lstset{language=JavaScript}
\begin{lstlisting}
app.post('/login', async (req, res) => {
  const username = req.body.username
  const password = req.body.password

  try {
      const result = await connection.execute(
          `SELECT *
            FROM dbahrim.app_users
            WHERE username = :username and userpassword = :password`,
          { username, password }
      );

      if (result.rows.length > 0) {
          res.end("Success");
      } else {
          res.end("Failure");
      }
  } catch (e) {
      res.end("Unknown error occured")
  }
})
\end{lstlisting}


\subsection{Bază de date}

Prin folosirea de proceduri parametrizate statice PL/SQL putem să prevenim SQL 
Injection întrucât cererea nu se va schimba, ci doar datele cu care sunt apelate.

Utilizarea de utilizatori ale bazelor de date către serverele care au privilegii limitate (de exemplu,
nu permit INSERT sau DELETE) vor limita cât de multe probleme un scrip de SQL Injection ar putea cauza. De exemplu, putem crea un utilizator care va fi folosit mereu 
de către aplicațiile integratoare pentru a comunica cu baza de date, iar acestui user îi oferim privilegii doar de SELECT.

\lstset{language=SQL}
\begin{lstlisting}
  CREATE USER new_user IDENTIFIED BY Password1234;
  GRANT SELECT ON APP_USERS TO new_user;
  GRANT SELECT ON APP_USERS TO new_user;
\end{lstlisting}

Limitarea conținutului la care un atacator ar avea acces prin criptarea datelor sensibile. De exemplu, criptarea parolelor, cardurilor, etc. 
Aceasta se poate face fie la nivel de server fie la nivel de bază de date, dacă baza de date suporta acest lucru.

Validarea datelor de intrare în PL/SQL poate fi folosită prin pachete precum
DBMS\_ASSERT care ne ajută să prevenim caracterele care ar putea cauza SQL Injection.

Întrucât procedurile stocate sunt apelate implicit cu drepturile creatorului, acestea
pot oferii privilegii ridicate pentru persoane care pot apela acele proceduri. Astfel, aceste proceduri ar trebui să fie apelate cu drepturile apelantului.


\section{Concluzie}

SQL Injection reprezintă una dintre cele mai grave vulnerabilități de securitate care poate compromite integritatea și confidențialitatea datelor stocate într-o bază de date, însă modul în care ne protejăm contra acestea este simplu, nu avem încredere 
în datele pe care le primim și încercăm în orice loc să le validăm sau să nu le folosim direct în query.




\printbibliography


\end{document}